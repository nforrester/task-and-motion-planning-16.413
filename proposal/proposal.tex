\documentclass{article}

\title{Advanced Lecture Proposal \\ 8.834/16.412}
\author{Neil Forrester \and Troy Astorino}


\begin{document}

\maketitle


\section{Abstract}
A primary task of modern robots is manipulating objects in their environment. A
prerequisite for manipulating something is knowing where it is. When most robots
look for things, their activities are limited to moving themselves, and
observing their environment with cameras (or other sensors). When people look
for things though, they open drawers, turn over rocks, and generally manipulate
their environment to facilitate their search. We discuss new research in how
robots can perform some of these same activities. In particular we focus on the
two tasks of deciding the probability that certain spaces contain certain
objects (based on how often objects occur together, and what objects will fit in
the space), and on deciding what order spaces should be searched in to maximize
the probability of finding desired objects as quickly as possible.

\section{Rough Outline}
\begin{enumerate}
\item Overview
\begin{enumerate}
\item Robotic object search has been limited to active visual search
\item Manipulation-based search requires an emphasis on planning as manipulation by
mobile robots is still error-prone and expensive
\item Observed information can inform a search
\begin{enumerate}
\item co-occurrence of objects
\item geometric information (which objects could fit into a given space)
\end{enumerate}
\end{enumerate}
\item Background research
\begin{enumerate}
\item Assumed functionality 
\begin{enumerate}
\item Execution of actions once they have been planned
\item Object recognition/computer vision
\end{enumerate}
\item Active visual search - list of papers
\end{enumerate}
\item Search planning
\item Co-occurrence based belief updates
\item Geometric constraint based belief updates
\item Results of paper
\item Future Directions - brief intro to generalized framework for integrated robot task and
motion planning
\end{enumerate}

\section{Division of Labor}
We’ll both work on all the slides (division of labor in the preparation stage
will be somewhat ad-hoc, according to time, ability, knowledge, etc). On
presentation day the current plan is that Neil Forrester will present the first
half (up through approximately part 4), and Troy Astorino will present the
second half (part 5 and onward). This is subject to change, of course, and
either of us may pop in to answer audience questions as appropriate.

\section{References}
\begin{itemize}
\item Wong, Lawson LS, Leslie Pack Kaelbling, and Tomás LozanoPérez. "Manipulation-based Active Search for Occluded Objects."
\item Kaelbling, Leslie Pack, and Tomas Lozano-Perez. "Integrated robot task and motion planning in belief space."
\item T. Kollar and N. Roy. “Utilizing object-object and object-scene context when planning to find things.”
\item L. P. Kaelbling and T. Lozano-Perez. “Unifying perception, estimation and action for mobile manipulation via belief space planning.”
\item Jean-Claude Latombe. “Motion Planning With Uncertainty: The Preimage Backchaining Approach”
\end{itemize}
\end{document}
